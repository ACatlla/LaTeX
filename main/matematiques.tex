\chapter{ Matemàtiques}

\section{Símbols matemàtics}

Símbol de sumatori $\sum_{inici}^{fi} eq.$ \\

Integral $\int_{inici}^{fi} eq.$\\

Fracció $\frac{numerador}{denominador}$\\

Símbol de la derivada $\partial$ \\

Símbol més, menys $\pm$

\section{Representació de funcions}
\subsection{Representació de funcions matemàtiques entre text}

Escribim text i entremig si volem posar una funció matemàtica $E=mc^2$ per exemple.

\subsection{Representació de les funcions matematiques independients}
\begin{itemize}
\item En cas de que volguem posar l'equació a part ho farem de la següent manera:
\begin{equation}
x=\frac{-b\pm\sqrt{b^2 - 4ac}}{2a}
\end{equation}
\item En cas de que volguem fer una resolució d'una equació
\begin{eqnarray*}
  \label{eq:fonamental}
  F(x) &=& x ( x^3 -2x - 1) \\
       &=& x^4 - 2x^2 - x \\
\end{eqnarray*}
\end{itemize}
