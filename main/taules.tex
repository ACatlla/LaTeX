\chapter{ Taules}
\section{Taules mòbils}
La taula corresponent és la taula \ref{tab:taula1} que es situa en la posició del text que correspon a la millor ubicació segons l'algoritme. 
És una taula mòbil centrada, que ajusta la seva amplada a l'amplada de la pàgina. Consta de tres columnes, la primera amb els elements tabulats a l'esquerre, la segona amb els elements tabulats al centre i la tercera amb els elements tabulats a la dreta. No consta de línies laterals. A l'apartat \ref{taules fixes} podrem veure taules fixes clicant a la referència.

\begin{table}[tbh]
  \centering
  \begin{tabular}{l|c|r}%primer elem tabulat a l'esquerra, segon tabulat al centre i tecer tabulat a la dreta i línies verticals entre mig
    \hline % línia horitzontal per separació
    \textbf{Capcelera c1}  & \textbf{Capcelera c3}  & \textbf{Capcelera de la c2} \\
    \hline
    element f1 c1  & element f1 c2  & element f1 c3 \\
    element f2 c1  & element f2 c2  & element f2 c3 \\
    \hline
  \end{tabular}
  \caption{Explicació de la taula}
  \label{tab:taula1}
\end{table}


\section{Taules fixes}
Taula amb quatre columnes tabulades a l'esquerre, les quals tenen una capcelera compartida entre dues columnes i la columna 4 no té capcelera.

\label{taules fixes}
\begin{center}
  \resizebox{\textwidth}{!}{ %
  \begin{tabular}{|l|l|l|l|}
    \hline
    \multicolumn{2}{|c|}{\textbf{Capcelera comú de la columna 1 i 2}} & \textbf{Capcelera de la columna3}& \\    
    \hline
    element f1 c1  & element f1 c2  & element f1 c3 & element f1 c4 \\
    element f2 c1  & element f2 c2  & element f2 c3 & element f2 c4 \\
    \hline
  \end{tabular}
  }
\end{center}
